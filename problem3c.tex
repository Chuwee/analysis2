Take the figure eight curve. We can split it into two curves cutting by the
intersection point, and then consider both curves separately. Traversing one of
the curves in any direction means we traverse the other in opposite direction.
By the deformation theorem, this reduces to calculating the integral caused by
the singularities at the interiors of both curves. Let's call the function
we're integrating $f(z)$. We can write the integral as

\[ \oint_\gamma \frac{{8z - 3}}{{z^2 - z}}  dz = \oint_\gamma f(z) dz\]

For the region on the left, denoted as $\gamma_1$, we can express $f(z)$ as:

\[ f(z) = \frac{{8z - 3}}{{z^2 - z}} = \frac{\frac{8z - 3}{z-1}}{{z}}\]

If we call $g_1(z) = \frac{8z - 3}{z-1}$, we can see that $g(z)$ is holomorphic
in the left region because it's the division of two entire functions and the
denominator is not 0 in this region. Thus, we can apply the Cauchy formula once
again.

\begin{align*}
    2\pi ig_1(z_0) & = \oint_{\gamma_1}\frac{g_1(z)}{z-z_0}dz
\end{align*}

Since $g_1(0) = 3$, $\oint_{\gamma_1}\frac{g(z)}{z-z_0}dz = 6\pi i$.

By a similar argument, call the right region $\gamma_2$ and $g_2(z) = \frac{8z
        - 3}{z}$, and we can see that $g(1) = 5$, so
$\oint_{\gamma_2}\frac{g(z)}{z-z_0}dz = 10\pi i$. Since we traverse the left
curve in the positive direction and the right curve in the negative direction,
we can add the two integrals together to get the value of the original
integral:

\begin{align*}
    \oint_\gamma \frac{{8z - 3}}{{z^2 - z}}  dz & = 6\pi i - 10\pi i = -4\pi i
\end{align*}