Since $f$ is entire, we can consider its MacLaurin series:
	      \begin{equation*}
		      f(z) = \sum_{n=0}^{\infty} \frac{f^{(n)}(0)}{n!} z^n =
		      a_0 +
		      a_1z +
		      a_2z^2 + \ldots
	      \end{equation*}
	      Where $a_i$ is the complex coefficient of $z^i$ in the MacLaurin
	      expansion. Assume $a_i \neq 0$ for $i \geq 2$, then $|f(z)| = a_0
		      +
		      a_1z +
		      a_2z^2 + \ldots$, so the limit becomes

	      \begin{equation*}
		      \lim_{|z| \to \infty}\frac{|f(z)|}{|z|^2} = \lim_{|z| \to
			      \infty}\frac{|a_0 + a_1z +
			      a_2z^2 + \ldots|}{|z|^2} = \lim_{|z| \to
			      \infty}\left|\frac{a_0 + a_1z +
			      a_2z^2 + \ldots}{z^2}\right|
	      \end{equation*}

	      Applying the division, we get:

	      \begin{equation*}
		      \lim_{|z| \to \infty}\left|\frac{a_0 + a_1z +
			      a_2z^2 + \ldots}{z^2}\right| = \lim_{|z| \to
			      \infty}\left|\frac{a_0}{z^2} + \frac{a_1}{z} +
		      a_2 + \ldots\right| = 0
	      \end{equation*}

	      This is true if and only if the complex number tends to 0, so we
	      have:

	      \begin{equation*}
		      \lim_{|z| \to \infty}\frac{a_0}{z^2} + \frac{a_1}{z} +
		      a_2 + \cdots = 0
	      \end{equation*}

	      And since the limit of the sum is the sum of the limits:

	      \begin{equation*}
		      \lim_{|z| \to \infty}\frac{a_0}{z^2} + \lim_{|z| \to
			      \infty}\frac{a_1}{z} +
		      \lim_{|z| \to \infty}a_2 + \cdots = 0
	      \end{equation*}

	      And we know that $\lim_{|z| \to
			      \infty}\left|\frac{a_0}{z^2}\right|
		      = 0$ and $\lim_{|z| \to \infty}\left|\frac{a_1}{z}\right|
		      =
		      0$, because $a_0$
	      and $a_1$ are constants. Once again, since the modulus is 0, the
	      number must be 0, so the limit becomes:

	      \begin{equation*}
		      \lim_{|z| \to \infty}a_2 + \lim_{|z| \to \infty}a_3z +
		      \cdots
		      = a_2 + \lim_{|z| \to \infty}\sum_{n=3}^{\infty}a_n
		      z^{n-2} =
		      0
	      \end{equation*}

	      So the limit of the sum is $-a_2$, like so:

	      \begin{equation*}
		      \lim_{|z| \to \infty}\sum_{n=3}^{\infty}a_n z^{n-2} =
		      -a_2
	      \end{equation*}

	      Taking the modulus of both sides, we get:

	      \begin{equation*}
		      \lim_{|z| \to \infty}\left|\sum_{n=3}^{\infty}a_n
		      z^{n-2}\right|
		      = |-a_2| = |a_2|
	      \end{equation*}

	      Since the LHS doesn't have singularities, and the limit is
	      finite,
	      we can bound the sum by a constant $M$:

	      \begin{equation*}
		      \left|\sum_{n=3}^{\infty}a_n z^{n-2}\right| \leq M
	      \end{equation*}

	      But, by Liouville's theorem ($*$), if $s(z) =
		      \sum_{n=3}^{\infty}a_n
		      z^{n-2}$ is
	      bounded, then it is constant. Since the
	      limit is $-a_2$, then

	      \begin{equation*}
		      s(z) = \sum_{n=3}^{\infty}a_n z^{n-2} = -a_2
	      \end{equation*}

	      But since $s(0) = 0$, then $-a_2 = 0$, so $a_2 = 0$, and $s(z) =
		      \sum_{n=3}^{\infty}a_n z^{n-2} = 0$

	      Going back to the Maclaurin expansion of $f$, we have:

	      \begin{equation*}
		      f(z) = a_0 + a_1z + a_2z^2 + z^2\sum_{n=3}^{\infty}a_n
		      z^{n-2} = a_0 + a_1z
	      \end{equation*}

	      If we rename $a_0 = a$ and $a_1 = b$, we get the final result:

	      \begin{align*}
		      f(z) = a + bz \\ a, b \in \mathbb{C}
	      \end{align*}

	      ($*$) Here we need to be careful as $s(z)$
	      was obtained by $\frac{f(z)-a_2z^2-a_1z-a_0}{z^2}$, so we would
	      really be
	      talking about a function which takes this fraction in $\C -
		      \{0\}$
	      and $0$
	      if $z = 0$, and this function is holomorphic at $\C - \{0\}$, but
	      we need to check that it is also holomorphic at $0$ to make sure
	      that we can apply the
	      theorem. Following the actual piecewise definition of $s(z)$

	      \begin{equation*}
		      s(z) = \begin{cases}
			      \frac{f(z)-a_2z^2-a_1z-a_0}{z^2} & \text{if } z
			      \neq 0
			      \\
			      0                                & \text{if } z =
			      0
		      \end{cases}
	      \end{equation*}

	      To check whether $s(z)$ is holomorphic at $0$, we need to check
	      if

	      \begin{equation*}
		      \lim_{h \to 0}\frac{s(h) - s(0)}{h-0} = \lim_{h \to
			      0}\frac{s(h)}{h}
	      \end{equation*}

	      converges.
	      Note $f(h) - a_2h^2 - a_1h - a_0 = h^2\sum_{n=3}^\infty a_n
		      h^{n-2}$, so

	      \begin{equation*}
		      \lim_{h \to 0}\frac{s(h)}{h} = \lim_{h \to
			      0}\frac{h^2\sum_{n=3}^\infty a_n h^{n-2}}{h^2} =
		      \lim_{h \to
			      0}\sum_{n=3}^\infty a_n h^{n-3} = a_3
	      \end{equation*}

	      So $s(z)$ is holomorphic at $0$, and thus it is entire.