\begin{enumerate}
	\item To describe $\Omega$ we first recall $\Log(z) =
		      \ln|z|
		      + i\Arg(z)$. Since $|z| < 1$ we have
	      $\ln|z| < 0$
	      and since $\Arg(z) \in
		      (-\pi/2, \pi/2)$. The logarithm $\ln|z|$
	      can take
	      any value in $(-\infty, 0)$, therefore we're
	      looking at
	      numbers of the form $x + iy$ where $x < 0$ and $y
		      \in
		      (-\pi/2, \pi/2)$.
	      \begin{equation*}
		      f(\Omega) = \{z \in \mathbb{C} : z = x +
		      iy, x <
		      0, y \in (-\pi/2, \pi/2)\}
	      \end{equation*}
	\item Similarly, we have $|z| > 1$ so $\ln|z| > 0$ and
	      $\Arg(z) \in [0, \pi]$. The logarithm $\ln|z|$ can
	      take any value in $(0, \infty)$, therefore we're
	      looking at numbers of the form $x + iy$ where $x >
		      0$
	      and
	      $y \in [0, \pi]$.
	      \begin{equation*}
		      f(\Omega) = \{z \in \mathbb{C} : z = x +
		      iy, x >
		      0, y \in [0, \pi]\}
	      \end{equation*}

	\item We're asked to find the branch cut of the function
	      $f(z)$. This means we must find where $\Arg(\Log\frac{z-i}{z+i})$
	      jumps from $-\pi$ to $\pi$. Arguing by the definition of the
	      logarithm, this means $\Arg(\ln|\frac{z-i}{z+i}| +
		      i\Arg(\frac{z-i}{z+i}))$ must jump from
	      $-\pi$ to $\pi$

	      For $\Arg(\ln|\frac{z-i}{z+i}| + i\Arg(\frac{z-i}{z+i}))$ to be
	      $-\pi$ or $\pi$, there
	      must be no imaginary part, so $\Arg(\frac{z-i}{z+i}) = 0$, which
	      means $\frac{z-i}{z+i}$ is
	      real. This then means that the fraction cancels out as a real
	      number, so $z$ must be purely imaginary (otherwise we're met with
	      $\frac{\Re(z)
			      + \Im(z) - i + i + i}{\Re(z) + \Im(z) - i} = 1 +
		      \frac{2i}{\Re(z) + \Im(z) -
			      i}$ which can't be real if $\Re(z) \neq 0$)
	      Let $z = yi$, then we're looking at points of the form
	      $\frac{yi-i}{yi+i} = \frac{y -1}{y+1}$. For this to be negative
	      (to be in the negative real axis), we have $-1 < y < 1$. Thus, the branch cut
	      is the imaginary axis from $-i$ to $i$ exclusive.

\end{enumerate}